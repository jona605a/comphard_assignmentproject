\section*{c) From decision variant to optimization variant}
%We now assume that we have a polynomial time algorithm for the decision problem. From this we will construct an polynomial time algorithm for the optimization problem. The pseudo code is shown below

We now assume that we have an algorithm for the decision problem where a call takes one computational step. From this we will construct an polynomial time algorithm for the optimization problem. The pseudo code is shown in algorithm \ref{alg:spP}. 

\begin{algorithm}[ht!]
\caption{Find the minimum mirror friendly spanning tree}\label{alg:spP}
\begin{algorithmic}
\State $B_{upper} \leftarrow \sum_{i=1}^m w(e_i)$
\State $B_{lower} \leftarrow 0$
\If{$A_d(G,B_{upper})$ = NO}
    \State return NO
\EndIf
\State $B \leftarrow \left\lfloor{\frac{B_{upper} + B_{lower}}{2}} \right\rfloor $
\While{true}
    \If{$A_d(G,B)$ = YES}
        \State $B_{upper} \leftarrow B$
    \Else
        \State $B_{lower} \leftarrow B+1$
    \EndIf
    \State $B \leftarrow \left\lfloor{\frac{B_{upper} + B_{lower}}{2}}\right\rfloor$
    \If{$B_{lower} = B_{upper}$}
        \State \textbf{break}
    \EndIf
\EndWhile
\For{$i$ in $1\dots m$}
    \If{$A_d((V,E\backslash \{e_i\}),B) = YES$}
        \State $E \leftarrow E \backslash \{e_i\}$ 
    \EndIf
    \If{$|E| = n-1$}
        \State return $E$
    \EndIf
\EndFor
\end{algorithmic}
\end{algorithm}

\subsection*{Correctness}
The first part of the algorithm simply binary searches for the correct $B$. After this step $B$ has the smallest possible value for our given $G$ and we now need to find the solution. To find the solution we use the same procedure as presented in the course for finding the solution for the MaximumClique problem.\\
We go through each edge and try to remove it. If $A_d$ still answers YES it means that there still exists a solution and we can remove this edge. If $A_d$ answers NO we keep the current edge. We will now argue that this always finds a correct solution. When $|E| = n-1$ we return our solution $E$. 
\subsubsection*{One solution}
We will first look at the case where there only exists one solution in $G$. A solution to this problem is spanning tree which is a set of edges $ST$ and in this case there only exists one such set. We now look at an edge $e$. If $e\in ST$ and we remove it there no longer exists a solution in $G$ and $A_d$ answers NO and we keep $e$. If $e\notin ST$ we still have $ST \subseteq E$ and the solution still exists in $G$. $A_d$ will therefore answer YES and we can remove $e$ and still have a solution. As $G$ is finite we at some point will have removed all edges not in $ST$. This final $G$ is our solution as it is a spanning tree with edge set $ST$.

\subsubsection*{Multiple solutions}
We now assume there are multiple solutions $ST_1\dots ST_k$ present in $G$. We now look at an edge $e$. If $e$ is an all $ST_i$ and we remove it then $G$ no longer has any solutions and $A_d$ answers NO so we keep $e$. Thus we never remove all solutions from $G$. If $e$ is present in at least $ST_j$ but not present in at least $ST_i$ then removing it does not destroy all solutions. Therefore $G$ will still have at least the solution $ST_i$. Then $A_d$ would answer YES and we would have destroyed at least $ST_j$ and thus decreased the number of solutions. If $e$ is present in no solution then we of course can just remove like the case with one solution. Since each all the solutions are different then each $ST_i$ has at least one edge not present in all the other solutions. Therefore while there are more than one solution present in $G$ there will always be an edge which does not destroy all solutions which we can remove. Doing this iteratively we at some point only have one solutions present in $G$ and we know from before that in this case we will find the correct solution.

\subsection*{Polynomial running time}
We will now argue that the algorithm presented above runs in polynomial time. 

Let $W=\sum_{i=1}^m w(e_i)$. First we binary search for the correct $B$ in the interval $[0,W]$. This takes $O(\log(W))$ time. However we know that $W \leq 2^{\|\mathbf{X}\|}$. So we get that this takes $O(\log(W)) = O(\log(2^{\| \mathbf{X}\|})) = O(\|\textbf{X}\|)$, so this step takes linear time. 
For each edge we some constant time work and make a call to $A_d$, which we assume takes constant time. As a result this step takes $O(m)$ time. Since each edge is specified in the input we know $m \leq \|\textbf{X}\|$. So we finally get a running time of $O(\|\textbf{X}\| + m) = O(\|\textbf{X}\|)$ which means the algorithm $A_o$ runs in polynomial time.
%For each edge we some constant time work and make a call to $A_d$, which we know is polynomial. As a result this step takes $O(m\cdot p(\|\textbf{X}\|))$. Since each edge is specified in the input we know $m \leq \|\textbf{X}\|$. So we finally get a running time of $O(\|\textbf{X}\| + \|\textbf{X}\|\cdot p(\|\textbf{X}\|)) = O(\|\textbf{X}\|\cdot p(\|\textbf{X}\|))$ and as the product of two polynomials is also a polynomial our algorithm runs in polynomial time.
